\section{Researching technique}

To identify common problems that arise for \textit{Scala on Android} developers it was necessary to collect data from people who already gained experience with the technology. In preparation for this task the first step was to identify an appropriate data-gathering technique, deciding between qualitative and quantitative research. The small size of the \textit{Scala on Android} community was a strong indicator for quality based research. Obtaining a reasonable sample size for quantitative researching methods would fail at the scope of the target group. Furthermore, qualitative research is a suitable technique when \enquote{investigators are interested in understanding the perceptions of participants} \cite[p. 72]{berg01} whereas quantitative methods are used when \enquote{researchers have clear ideas about the type of information they want to access and about the purpose and aims of their research} \cite[p. 72]{berg01}. Therefore qualitative research is technique of choice for the purpose of revealing difficulties in dealing with the \textit{Scala} programming language on the \textit{Android} platform.