\section{Conclusion and Feedback}

Creating a comprehensive documentation for \textit{Scala on Android} was a challenging exercise. I managed to address every issue which was discovered by the help of the interviews, but in some rare cases I saw no other solution than to surrender and accept that there was currently no way to fix this particular problem. These documentation gaps were a disappointing experience and made it incredibly difficult for me to still bring myself to publishing the document, yet alone asking the developers for feedback explicitly.

Luckily both, the former interviewees and the community as a whole, appeared to appreciate the website. After setting the documentation up and publishing it on the web, I reached out to the interview partners asking them for their thoughts about my work result.

\begin{displayquote}

	Hi \underline{\hspace{2cm}},\\\\

	I just published the current state of my efforts for the \textit{Scala on Android} documentation here on Reddit. But since you shaped the form of the project through our interview I am especially interested in your thoughts about it.
	It's far from perfect and turned out to be way more work than I expected, but I'd still like you to have a look at it. And, if time permits, provide me with brief feedback (3-4 sentences suffice) of what you like and don't like about it.\\\\
	Thank you,\\
	Nik

\end{displayquote}

\hrulefill

\begin{displayquote}

	You really did a great job there. Especially the hello-scala project is a huge win for me. I can't believe why we didn't have anything like this before. Maybe you should put more emphasize on collaboration to keep the project active? It'd be a shame if it was a wasteland in a couple of months. I'll try to stick around and make a contribution every now and then so that we can keep it up to date.

\end{displayquote}

\hrulefill

\begin{displayquote}

	I'm pleasantly surprised by the overall appearance, it looks very clean and elegant. Your articles seem well thought out and I've enjoyed reading so far. The chapter about parcelables was especially useful for me. I always kinda avoided that with Scala (and hated it with Java). Incredibly good job. Will definitely recommend!

\end{displayquote}

\hrulefill

\begin{displayquote}

	I think you did good work in that short time. I did not read all of it now but looks very good. Too bad you couldn't add gradle. Maybe I should try out sbt ;)

\end{displayquote}

Unfortunately, not all interviewees responded to my inquiry but other community members that did not participate in the interviews also left some valuable and encouraging feedback when I promoted the website on their platform.

\begin{displayquote}

	This [is] one of those things I've been looking for, for a while. \\\\
	Please, can you keep updating? \\\\
	I on the site would emphasize that others could contribute on Github more prominently; maybe it will help growth. \\\\
	What can I, Scala and Android noob, do to help?

\end{displayquote}

\hrulefill

\begin{displayquote}

	This is great work. Thanks so much.

\end{displayquote}

\hrulefill

\begin{displayquote}

	I haven't tried to follow the steps but I glanced through the whole thing and it seems like a great start!

\end{displayquote}

\hrulefill

\begin{displayquote}

	I am very pleased to bookmark this for perusal as it is something I want to learn very soon -- in fact, I'm only delaying so I can brush up on my Scala in the first place. I hope to look at it in the near future!

\end{displayquote}

It is still to be clarified whether I spent too much time on making the documentation fun, rather than improving its content.

\begin{displayquote}

	I spent more time playing with your \texttt{a:hover} css effect than actually reading the documentation. I am easily amused :/

\end{displayquote}

Since I received this overall positive feedback I am quite excited about the project and looking forward to refining it. I'm glad that I has the possibility to create such a valuable resource for the development community as part of my thesis.