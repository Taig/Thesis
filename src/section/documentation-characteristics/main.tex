\section{Outlining the documentation characteristics}

With the intention to create a documentation in the form of a publicly available website, the opportunity occurs to go way beyond static, textual content. A website allows to add features that have the potential to enhance the overall reading an learning experience. Listed below are the features and characteristics which I considered important in order to create a valuable learning resource.

\begin{description}

	\item[Content]\hfill

	It is easy to give in to the temptation of developing yet another great feature. But however the development of the documentation proceeds, the actual, textual content has to always be the number one priority.

	\item[Progressive enhancement]\hfill

	Every feature on top of the basic content has to be implemented according to the strategy of \textit{progressive enhancement}. Meaning that a feature that requires the client's browser to have a certain ability (e.g. executing \textit{JavaScript} code) should never have a negative impact on browsers that are not able to fulfill the requirement (besides missing out on the feature in question).

	As an illustrative example of this strategy, imagine a timezone specific \textit{DateTime} value embedded into the \ac{HTML}, such as \textit{Wed Jun 24 17:48:48 2015 +0200}. To improve readability, the \textit{JavaScript} \ac{API} \textit{Date.toLocaleString()} is used to convert the time according to the user's timezone (e.g. \textit{24.6.2015, 17:48:48}). If the client's \textit{JavaScript} engine did not implement this function, or the client does not run \textit{JavaScript} at all, the raw time information must still be accessible.

	\item[Static]\hfill

	The content has to be delivered as static \ac{HTML} files and should not require any serverside processing. This simplifies hosting, improves response times and makes it easier to edit the sources.

	\item[\ac{AJAX}]\hfill

	To fulfill the \textit{progressive enhancement} and \textit{static} requirements, the document should refrain from loading crucial content dynamically via \ac{AJAX}. This way the website is also more accessible to search engine bots that analyze the content. They are at risk to miss out dynamic content.

	\item[Semantics]\hfill

	Not only the text, but also the underlying \ac{HTML} structure should semantically make sense. This is necessary to increase accessibility for screen readers and does also improve the understanding of the website for search engine bots. This requirement extends to good practices, such as always providing meaningful \textit{alt} tags for images as well as using \textit{title} tags to provide contextual information that may improve readability.

	\item[Collaboration]\hfill

	Since the contained information are always at risk to become outdated rather quickly, the document should be manageable in a highly collaborative way allowing any reader to easily propose or make changes to the document. This does of course also apply for fixing misspellings or substantial errors, improving the wording or even the layout.

	\item[Readability]\hfill

	Textual content has to be presented in such a way that it is pleasantly readable. This relates especially to fonts, their sizing, line spacing, but also the overall contrast. Furthermore, the user should be able to increase the font size without breaking the website or making content inaccessible.

	\item[Gateway]\hfill

	There are plenty of good blog posts and documents scattered across the web for nearly all the mentioned problems. The website has to provide sufficient information to solve the user's issue but also point him the way to more detailed information or to necessary perquisites that might be taken care of first.

	\item[Responsive design]\hfill

	The website should render well, independently of the readers screen dimension or density.

	\item[Navigation]\hfill

	It has to be easy to advance through the pages. Skipping to the next or previous chapter should be a prominent option, but also navigating to every other topic of the documentation.

	\item[SEO]\hfill

	Every page has to contain meta tags and meaningful titles embedded into the document. A good search engine ranking does not only increase visibility on the web, but with proper meta information it also allows the user to jump immediately to the content he cares about right from the search results.

	\item[Syntax highlighting]\hfill

	Programming code that is embedded into the documentation should be easily identifiable as such. It has to be rendered with a monospace font and should feature syntax highlighting applied on top of that.

	\item[Examples]\hfill

	The content should be enriched with practical examples of the described topics wherever possible.

	\item[Indicate outdated information]\hfill

	Information that are at risk to become outdated must be highlighted as such. If the user finds himself in a situation where he is unable to recreate given instructions he should first consider that the given information might no longer be valid rather than searching for mistakes in his project. In addition to that, the user should be able to see when was the last time a paragraph has been updated.

	\item[Fun]\hfill

	Last but not least, it may never be frustrating to read and use the website. It must instead be fun and beautiful to look at.

\end{description}