\section{Introduction}

\textit{Scala} is a programming language for the \acf{JVM} which \enquote{fuses object-oriented and functional programming} \cite[p. 1]{odersky04} paradigms. It has plenty of similarities to the better known \textit{Java} programming language and can seamlessly interact with its code \cite[p. 2]{odersky04}. But with \textit{Scala's} bias towards immutable data, a uniform object model (every value is an object) \cite[p. 3]{odersky04} and pattern matching \cite[p. 13]{odersky04}, it allows the developer to be more expressive, leading to more concise code.

Applications for the \textit{Android} platform are written in \textit{Java} and do then get compiled to \ac{JVM} byte-code. In a next step the byte-code is converted to an \textit{Android}-specific format and is now ready for execution on the \acf{ART}, \textit{Android's} \ac{JVM} implementation. Since compiled \textit{Scala} code is also ordinary \ac{JVM} byte-code, the thought of converting it to \textit{Android} stands to reason. The idea of bringing the advantages of \textit{Scala} to the \textit{Android} platform has already led to a variety of community developed tools that simplify this process, making it a considerable alternative to \textit{Java}. Unfortunately, substituting the technologies is still not as easy as one might hope. It does in fact introduce some obstacles which are hard, but not impossible, to overcome.

Getting \textit{Scala on Android} to work is primarily a matter of configuration which is unsatisfactorily documented. To make the procedure more accessible, this thesis is dedicated to creating the missing documentation with the intention to make the technology worth considering for a broader audience.