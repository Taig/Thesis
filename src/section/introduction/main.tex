\section{Introduction}

The motivation to create Scala applications for the Android platform is as simple as the drive to create the Scala 
programming language in the first place. Java, as a language, has the reputation to be verbose and restrictive leaving 
plenty room for improvement. Scala on the contrary is a more modern language which is rich on features, combines 
paradigms of functional programming with object-orientated programming and has a bias towards immutable variable states
instead of mutable variable states. Just like Java applications, executable Scala programs usually run on the \ac{JVM}
because the source code gets compiled to ordinary Java bytecode. Furthermore the language maintains compatibility to
Java, allowing it to benefit from numerous tools and libraries that are available in the Java ecosystem.

This makes Scala a perfect candidate for development on Android because it produces bytecode which can be processed by
the Android toolchain and is also able to interact with the Android \ac{SDK} as it is compatible to the official Java
\ac{API}. Bringing the advanced features of Scala to the popular Android platform is a promising undertaking which pays
off, once the initial difficulties have been overcome.