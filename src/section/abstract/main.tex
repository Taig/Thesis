\section*{\centering{Abstract}}

Android applications are usually developed with the Java programming language and the Android Software Development Kit
(SDK). The program code is compiled to Java bytecode and then translated to so-called Dalvik bytecode, an
Android-specific bytecode format, which is executed by the Android Runtime (ART). Although this is the only officially
supported and documented way of creating SDK apps, Android’s compatibility with Java bytecode opens the door to a
variety of other Java Virtual Machine (JVM) technologies. In fact every programming language that can be compiled to
Java bytecode and which is also able to interact with the Java Android SDK, might be easily used as a fully functional
alternative to Java.

The Scala programming language is one out of several JVM technologies that fulfills these requirements. On top of that
it covers most features of Java and adds a variety of useful tools, such as: type inference, functional programing
paradigms, actor-based concurrency and multiple inheritance with traits (as an improvement to Java interfaces) to just
mention a few of them. Its steadily growing community has recently started a couple of ambitious projects aiming to get
rid of possible starting difficulties which keep the majority of Scala developers from conquering Android.

Liefery is a startup that is about to shift the concept of a bike courier from city level to national and international
deliveries. And instead of only relying on staff to perform deliveries, Liefery involves every Android user in the
distribution process. By an app which allows its users to sign up as a distributor, one is able to receive delivery job
offers, inform the recipient about the progress and to finally complete the order by obtaining the customer’s signature
— everything within the app, written in Scala!