\section*{\centering{Abstract}}

\textit{Android} applications are usually developed with the \textit{Java} programming language and the \textit{Android} \ac{SDK}. The program code is compiled to \textit{Java} byte-code and then translated to so-called \textit{Dalvik} byte-code, an \textit{Android}-specific format, which can now be executed by the \ac{ART}. Although this is the only officially supported way of creating \ac{SDK} applications, \textit{Android's} transitive compatibility with \textit{Java} byte-code opens the door to a variety of other \ac{JVM} technologies. In fact, every programming language that can be compiled to \textit{Java} byte-code and which is also able to interact with the \textit{Android} \ac{SDK}, might be used as a fully functional alternative to \textit{Java}.

The \textit{Scala} programming language is one \ac{JVM} technologies that fulfills these requirements. It is fully interopable with \textit{Java} \acp{API} and has the potential to bring functional programming paradigms to the \textit{Android} platform. Unfortunately, getting started with \textit{Scala on Android} usually turns out to be an error-prone and frustrating process. To get it running, the user needs a deep understanding of the build process to configure it in such a way that it is sophisticated enough to result in a valid build.

In order to make this process more accessible, as part of this Bachelor's thesis, a web-based documentation will be created that offers a comprehensive \textit{Scala on Android} guide. To identify obstacles that developers are facing with this technology, it is first necessary to collect data with the means of qualitative research.