\section{Identifying the problems}

In order to create a documentation that contains the answers necessary to make \textit{Scala on Android} a smooth experience, it was first necessary to discover the questions. Having previously worked with \textit{Scala on Android} intensively, I gained a rough overview of the difficulties that a developer faces. To broaden this picture it was necessary to collect data of developers that also made experiences in this field.

\subsection{Researching technique}

In preparation for this task the first step was to identify an appropriate data-gathering technique, deciding between qualitative and quantitative research. The small size of the \textit{Scala on Android} community was a strong indicator for quality based research. Obtaining a reasonable sample size for quantitative researching methods would fail at the scope of the potential target group. Furthermore, qualitative research is a suitable technique when \enquote{investigators are interested in understanding the perceptions of participants} \cite[p. 72]{berg01} whereas quantitative methods are used when \enquote{researchers have clear ideas about the type of information they want to access and about the purpose and aims of their research} \cite[p. 72]{berg01}. Therefore qualitative research is the technique of choice for the purpose of revealing difficulties in dealing with the \textit{Scala} programming language on the \textit{Android} platform.

Since the development community is scattered across the globe I had to find ways to find and reach out to potential peers to then conduct a remote interview.

\subsection{Finding interview partners}

I planned to promote my intentions on relevant community platforms (such as \textit{reddit/r/scala}, the \textit{Scala} and the \textit{Scala on Android} mailing lists) hoping to reach a diverse target group. If this approach failed I intended to contact developers that were actively committing in this area on \textit{GitHub} directly.

My postings on the community portals enjoyed a surprisingly great popularity and enabled me to win 5 individuals over, willing to tell me about their development experiences.

\subsection{Preparing the interviews}

\subsubsection{Interview outline}

As explained in \textit{Qualitative research methods for the social sciences} I started by creating an interview \textit{outline} which \enquote{lists all the broad categories} \cite[p. 72]{berg01} that are relevant to the study. The outline helped me to get a better idea of the information I wanted to obtain and established a basis where I could afterwards derive the actual interview questions from.

When talking to developers about \textit{Scala on Android} I planned to collect data about the following topics (with a brief explanation of the particular purpose):

\begin{description}

	\item[Background in software development]\hfill \\

	Getting an idea of the development stacks the interviewee is comfortable with and what let to the decision of giving \textit{Scala on Android} a try

	\item[Experiences with Scala and Android]\hfill \\

	Finding out if and how the technologies were used before they were brought together on the \textit{Android} platform with the intention to get a clear picture about the importance of prior knowledge

	\item[Experiences with Scala on Android]\hfill \\

	The core of the conversation; which tools have been used, what were the use cases, as well as the overall satisfaction and experiences in detail

	\item[Opinions on Scala vs. Java on Android]\hfill \\

	Retrospectively, get the interviewee's evaluation whether switching to \textit{Scala} was worth the hassle and if he would stick to the technology in future projects

\end{description}

\subsubsection{Interview types}

There are numerous classifications for the structure of an interview available, however L. Berg identified the \textit{standardized}, the \textit{unstandardized} and the \textit{semistandardized} interview types \cite[p. 68]{berg01} to which I refer to hereafter.

The different interview types vary primarily in the strictness of their schedule. A standardized interview is formally structured and offers \enquote{each subject approximately the same stimulus so that the responses to the questions, ideally, will be comparable} \cite[p. 69]{berg01}. It is suitable for studies where researchers \enquote{have fairly solid ideas about the things they want to uncover during the interview} \cite[p. 69]{berg01} while leaving little room to discover anything beyond that.

In contrast to the standardized interview, the unstandardized alternative does not rely on a questions schedule at all. The idea behind this strategy is that it is impossible to \enquote{know in advance what all the necessary questions are} \cite[p. 70]{berg01} and that the interviewer must instead come up with appropriate follow-up questions in the particular situations. This approach is useful to reveal information \enquote{when researchers are unfamiliar with respondents' life styles [...] or customs} \cite[p. 70]{berg01}.

A semistandardized interview is located somewhere between the strictly scheduled and completely unscheduled interview types. A researcher who conducts a semistandardized interview is equipped with a rough question schedule and has the freedom to digress and to go deeper into certain topics if he expects to gain further information which may be valuable to his research.

I settled with the semistandardized interviewing technique for my research. A standardized interview seemed inappropriate as I aimed to learn about different perceptions and to gather insights which I missed during my exposure to \textit{Scala on Android}. Furthermore, I wanted to rely on some sort of schedule because I did not conduct interviews before and felt uncomfortable with the idea of coming up with ad-hoc questions as practiced in an unstandardized interview.