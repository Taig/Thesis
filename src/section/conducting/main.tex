\section{Conducting the interviews}

The five interviewee partners which I recruited on the community platforms were located in California (U.S.), China and Estonia. Due to the significant time differences to these locations it was very difficult to settle an appointment. I therefore had to conduct the U.S. and China interviews at night time. All of the interviews where carried out via \textit{Skype} calls. I left the interviewee the option to decide whether he preferred to have a video or just a voice conversation, but all without exception favored the voice call. As a consequence of this, the interviews went off as an ordinary telephone interview, missing the opportunity to catch non-verbal cues \cite[p. 82]{berg01}.

Getting know to each other and talking about technology was something the respondents seemed very eager about. In fact, once we started talking about the second schedule point, \textit{Scala on Android}, our conversation drifted into an unstandardized interview right away. Most of my questions were already answered by the detailed stories I got to hear. I then fell back to asking follow-up questions in order to deepen topics where appropriate. Despite me thinking that asking the interviewee for build configurations or source code snippets might go too far, my peers were happy to share those with me when I appeared to be interested.

All in all, none of the interviews could be conducted within the intended time frame of one hour. But since the conversations did not feel compulsive, my share of the talking was little and the gained information extremely valuable, I interpreted this as a good sign and did not attempt to shorten the conversations \cite[p. 81]{berg01}.

The sample group appeared very excited about the intentions of my research and acknowledged that a well-engineered \textit{Scala on Android} documentation would be a valuable resource for their own work and could also have a major impact on the growth of the community.