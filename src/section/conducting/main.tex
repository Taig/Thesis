\section{Conducting the interviews}

The five interview partners which I recruited on the community platforms were located in the United States, the United Kingdom, Estonia and China. Due to the significant time differences to some of these locations it was particularly difficult to settle an appointment. I therefore had to conduct the U.S. and China interviews at night time. All of the conversations were carried out via \textit{Skype} calls. I left the interviewees the option to decide whether they preferred to have a video or just a voice call, but all without exception favored the voice variant. As a consequence of this the interviews went off as ordinary telephone interviews, missing the opportunity to catch non-verbal cues \cite[p. 82]{berg01}.

Getting to know each other and talking about technology was something the respondents seemed very eager about. In fact, once we started talking about the second schedule point, \textit{Scala on Android}, conversations tended to drift into an unstandardized interview. Most of my questions were already answered by the detailed stories I got to hear. In this situation I fell back on asking follow-up questions in order to deepen topics when appropriate. Despite me thinking that asking the interviewee for build configurations or source code snippets might go too far, my peers were happy to share those with me when I appeared to be interested.

All in all, none of the interviews could be conducted within the intended time frame of one hour. But since the conversations did not feel compulsive, my share of the talking was little and the gained information extremely valuable, I interpreted this as a good sign and did not attempt to shorten the conversations \cite[p. 81]{berg01}.

The sample group appeared excited about the intentions of my research and acknowledged that a well-engineered \textit{Scala on Android} documentation would be a valuable resource for their own work and could also have a major impact on the growth of the community.